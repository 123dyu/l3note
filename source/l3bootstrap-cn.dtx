% \iffalse meta-comment
%
%% File: l3bootstrap-cn.dtx
%
% Copyright (C) 2023 by Yu Du <3531243657@qq.com>
% 
% This work may be distributed and/or modified under the
% conditions of the LaTeX Project Public License, either
% version 1.3c of this license or (at your option) any later
% version. The latest version of this license is in:
% 
%   http://www.latex-project.org/lppl.txt
% 
% and version 1.3 or later is part of all distributions of
% LaTeX version 2005/12/01 or later. 
%
%<*driver>
\documentclass[full,kernel]{l3doc}
\usepackage{ctex}
\usepackage{cprotect}
\usepackage{listings}
\usepackage{tcolorbox}
\tcbuselibrary{listings,breakable}

\lstdefinestyle{base}{
  gobble = 2,% 空两格再排版
  % escapechar = *,% *所包含的内容会按LaTeX格式排版
  escapeinside = {(*}{*)},
  columns = fullflexible,% 设置字符列为非等宽
  basicstyle = \small\ttfamily,% 整体格式设置
  keywordstyle = \bfseries\color{blue!60!black},% 关键字格式
  commentstyle = \itshape,
  backgroundcolor = \color{gray!6},% 背景颜色
  breaklines = true,% 自动换行
  frameround = ttff,% 边框四角弧度,t则弧
  frame = leftline,
  framerule = 1.6pt,% 边框线宽度
  framesep = 0pt,% 边距
  xleftmargin=1.5em,
  framexleftmargin = 1em
}

\lstdefinestyle{shell}{
  style = base,
  rulecolor = \color{red!70!black},
  language = bash,
  alsoletter = {-}
}

\lstdefinestyle{latex}{
  style = base,
  language = [LaTeX]TeX,
  alsoletter = {*, -}
}

% shell和LaTeX代码环境的排版
\lstnewenvironment{shell}[1][]{
    \lstset{style=shell, #1}
}{}

\lstnewenvironment{latexsyntax}[1][]{
    \lstset{style=latex, rulecolor=\color{cyan}, #1}
}{}

\lstnewenvironment{latexexample}[1][]{
    \lstset{style=latex, rulecolor=\color{teal}, #1}
}{}


% 代码盒子
\newtcblisting{tcblatexexample}{
  colback = black!1!white,colframe = blue!60!black,breakable,
  listing options = {
    breaklines = true,
    basicstyle = \small\ttfamily,%整体格式设置
    columns = fullflexible,%设置字符列为非等宽
  }
}

\ExplSyntaxOn
\legacy_if_set_true:n{bootstrap}
\ExplSyntaxOff
\begin{document}
  \DocInput{\jobname.dtx}
\end{document}
%</driver>
% \fi
%
% \title{^^A
%    \pkg{l3bootstrap}宏包——\textbf{导引代码(Bootstrap code)}^^A
% }
%
% \author{^^A
%  \LaTeX{}项目组【著】\thanks
%    {^^A
%      E-mail:
%        \href{mailto:latex-team@latex-project.org}
%          {latex-team@latex-project.org}^^A
%    }\\
%  dyu【译】\thanks{E-mail: 3531243657@qq.com}
% }
%
% \date{分发于2023-11-09}
%
% \ifbootstrap
% \maketitle
% \fi
%
% \begin{documentation}
% 
%
% \ifbootstrap
% \section{使用\LaTeX3模块}
%
% \LaTeX3模块被设计在\LaTeXe{}上使用,该模块自\LaTeXe{} 2020-02-02以来已经预
% 加载,而对于较旧的格式,仍可使用|\usepackage{expl3}|或|\RequirePackage{expl3}| 
% 来将其整体载入。
%
% \LaTeX3有一套不同于标准\LaTeXe{}的编程语法,下述函数用于构建\LaTeX3编程环境。
% \fi
%
% \begin{function}[updated = 2011-08-13]{\ExplSyntaxOn, \ExplSyntaxOff}
%   \begin{syntax}
%     \cs{ExplSyntaxOn} \meta{\LaTeX3代码} \cs{ExplSyntaxOff}
%   \end{syntax}
%   |\ExplSyntaxOn|和|\ExplSyntaxOff|分别用来开启和关闭\LaTeX3编程环境。比如在
%   \file{.tex}文件中,可以这样使用它们:
%   \begin{latexsyntax}[gobble=6]
%     \documentclass{<文类文件基本名>}
%     % 如今(自2020-02-02之后)可以不用载入expl3宏包就能使用(*\LaTeX3*)语法了
%     \begin{document}
%     % 显示在PDF文件中的内容
%     \ExplSyntaxOn % 开启(*\LaTeX3*)编程环境
%     ... 
%     \ExplSyntaxOff % 关闭(*\LaTeX3*)编程环境
%     \end{document}
%   \end{latexsyntax}
% \end{function}
% 
% \begin{function}[updated = 2017-03-19, updated = 2023-08-03]
%   {\ProvidesExplPackage, \ProvidesExplClass, \ProvidesExplFile}
%   \begin{syntax}
%     \cs{RequirePackage}|{expl3}|
%     \cs{ProvidesExplPackage} \Arg{宏包文件基本名} \Arg{发行日期} \Arg{版本号} \Arg{描述}
%     \cs{ProvidesExplClass} \Arg{文类文件基本名} \Arg{发行日期} \Arg{版本号} \Arg{描述}
%     \cs{ProvidesExplFile} \Arg{其他文件基本名} \Arg{发行日期} \Arg{版本号} \Arg{描述}
%   \end{syntax}
%   用于标识当前文件是一个宏包/文档类/其他文件(也就是除文档类/宏包之外的任何文件,
%   通常使用\tn{input}命令来载入它们),同时提供该文件的相关(发行)信息(包括:
%   文件名、发行日期、版本号和文件描述)。
% \end{function}
% 
% 如果要编写宏包或文档类,按照如下方式(与\LaTeXe{}的宏包和文档类编写规范类似
% \footnote{具体可参看:\href{https://github.com/CTeX-org/ctex-doc/tree/master/clsguide-zh-cn}{《写给\LaTeXe{}类与宏包的作者(中译版)》}})来操作方可进入\LaTeX3编程环境:
%
% \begin{latexsyntax}[gobble=4]
%   \ProvidesExplPackage {<宏包名>} {<发行日期>} {<版本号>} {<描述>}
%   % 文档类则使用(*\cs{ProvidesExplClass}*)
%   % 其他文件(文档类/宏包之外的任何文件)则使用(*\cs{ProvidesExplFile}*)
%   % 之后开启(*\LaTeX3*)编程环境(注意此后@也被视为字母),文末自动关闭(下划线_的类别码恢复为8,冒号:和@的类别码恢复为12)
% \end{latexsyntax}
% 其中,\meta{发行日期}的格式应为:\meta{年}/\meta{月}/\meta{日} 或ISO日期格式:\meta{年}-\meta{月}-\meta{日}。
% 对于\meta{版本号}的格式而言,在书写时以小写字母“v”作为前缀的版本标识符是可选的(可有可无),如果给出的\meta{版本号}是不以v开头的“纯”字符串,那么系统会自动在其前加一个v,比如:
% \begin{latexexample}[gobble=4]
%   % 下面两行代码:
%   \ProvidesExplPackage{test}{1.0.0}{2023-08-03}{Description}
%   \ProvidesExplPackage{test}{v1.0.0}{2023-08-03}{Description}
%   % 各自在编译时都会在屏幕和.log 文件输出类似下面这样的信息:
%   Package: test 2023-08-03 v1.0.0 Description
%   % 即前导符“v”是可选的
% \end{latexexample}
% 
% 如果不想在上述命令中直接书写文件(宏包/文档类/其他文件)的相关(发行)信息(包括:文件名、版本号、发行日期和文件描述),那么可以使用如下命令来提取这些信息:
% 
% \begin{function}[updated = 2012-06-04]{\GetIdInfo}
%   \begin{syntax}
%     \cs{GetIdInfo} |$Id:| \meta{SVN字段信息} |$| \Arg{描述}
%     \meta{SVN字段信息}的格式为:\meta{文件基本名}.\meta{文件扩展名} \meta{版本号} \meta{发行日期} \meta{其他信息}
%     \meta{版本号}格式:不加前导符v
%     \meta{发行日期}格式为:\meta{年}-\meta{月}-\meta{日}
%   \end{syntax}
%   用于提取文件(宏包/文档类/其他文件)的相关(发行)信息,注意在|$...$|之间的空格不会被忽略。
%   当按规定形式(|$Id:| \meta{文件基本名}.\meta{文件扩展名} \meta{版本号} \meta{年}-\meta{月}-\meta{日} \meta{其他信息} |$| \Arg{描述})填写完文件的相关(发行)信息后,这些信息就会被存储在以下命令中:
% \end{function}
% \begin{description}
%   \item[\cs{ExplFileName}:] 文件基本名
%   \item[\cs{ExplFileExtension}:] 文件扩展名
%   \item[\cs{ExplFileDate}:] 发行日期
%   \item[\cs{ExplFileVersion}:] 版本号 
%   \item[\cs{ExplFileDescription}:] 文件描述
% \end{description}
% 之后便可在\cs[no-index]{ProvidesExpl...}系列命令中用上述文件信息命令来代替原本需要直接书写的信息,比如:
% \begin{latexexample}[gobble=4]
%   \ProvidesExplPackage{\ExplFileName}
%     {\ExplFileDate}{\ExplFileVersion}{\ExplFileDescription}
% \end{latexexample}
% 
% 这一函数的定义如下:
% 
% \begin{latexsyntax}[rulecolor=\color{yellow},gobble=4]
%   \cs_new_protected:Npn \GetIdInfo
%     {
%       \tl_clear_new:N \ExplFileDescription
%       \tl_clear_new:N \ExplFileDate
%       \tl_clear_new:N \ExplFileName
%       \tl_clear_new:N \ExplFileExtension
%       \tl_clear_new:N \ExplFileVersion
%       \group_begin:
%       \char_set_catcode_space:n { 32 }
%       \exp_after:wN
%       \group_end:
%       \__file_id_info_auxi:w
%     }
%   \cs_new_protected:Npn \__file_id_info_auxi:w $ #1 $ #2
%     {
%       \tl_set:Nn \ExplFileDescription {#2}
%       \str_if_eq:nnTF {#1} { Id }
%         {
%           \tl_set:Nn \ExplFileDate { 0000/00/00 }
%           \tl_set:Nn \ExplFileName { [unknown] }
%           \tl_set:Nn \ExplFileExtension { [unknown~extension] }
%           \tl_set:Nn \ExplFileVersion {-1}
%         }
%         { \__file_id_info_auxii:w #1 ~ \s__file_stop }
%     }
%   \cs_new_protected:Npn \__file_id_info_auxii:w
%       #1 ~ #2.#3 ~ #4 ~ #5 ~ #6 \s__file_stop
%     {
%       \tl_set:Nn \ExplFileName {#2}
%       \tl_set:Nn \ExplFileExtension {#3}
%       \tl_set:Nn \ExplFileVersion {#4}
%       \str_if_eq:nnTF {#4} {-1}
%         { \tl_set:Nn \ExplFileDate { 0000/00/00 } }
%         { \__file_id_info_auxiii:w #5 - 0 - 0 - \s__file_stop }
%     }
%   \cs_new_protected:Npn \__file_id_info_auxiii:w #1 - #2 - #3 - #4 \s__file_stop
%     { \tl_set:Nn \ExplFileDate { #1/#2/#3 } }
% \end{latexsyntax}
% 从上述定义中可以看出:
% 
% \begin{enumerate}
%   \item 在|$...$|中,以空格来分隔参数,比如“|ab|”表示一个参数,而“\verb*|a b|”则表示两个参数。另外,在\meta{SVN字段信息}中,\meta{文件基本名}、\meta{文件扩展名} 、\meta{版本号}和\meta{发行日期}必须按规定格式正确填写,它们会关系到\cs{ExplFileName}、\cs{ExplFileExtension}、\cs{ExplFileVersion}、\cs{ExplFileDate} 这几个命令的取值;而\meta{其他信息}则可随意填写,比如可以是作者信息(作者的姓名、邮箱等)。举例说明如下:
%   \begin{latexexample}[gobble=6]
%     % 使用如下代码:
%     \GetIdInfo $Id: test.dtx 1.0.0 2023-09-03 11:00:00Z latexstudio <latexstudio@qq.com> $ {XXX Theme for the Beamer class}
%     % 查看文件信息命令:
%     \par \ExplFileName 
%     \par \ExplFileExtension
%     \par \ExplFileVersion
%     \par \ExplFileDate 
%     \par \ExplFileDescription
%     % 将会得到:
%     test
%     dtx
%     1.0.0
%     2023/09/03
%     XXX Theme for the Beamer class
%     % (*\kaishu 注意:该例中\meta{其他信息}所包含的内容(也就是在\meta{发行日期}之后和 \verb|$| 之前的所有内容)为:|11:00:00Z latexstudio <latexstudio@qq.com> |,不要将 |11:00:00Z|(即\meta{时}:\meta{分}:\meta{秒}Z)误以为是\meta{发行时间}所包含的内容。*)
%   \end{latexexample}
% 
%   \item 如果|$...$|之间只出现“Id”(不包括其后的冒号“:”),即|$Id$|,那么文件信息命令的值将分别为:
%   \begin{itemize}
%     \item \cs{ExplFileName}: { [unknown] }
%     \item \cs{ExplFileExtension}: { [unknown~extension] }
%     \item \cs{ExplFileVersion}: {-1}
%     \item \cs{ExplFileDate}: { 0000/00/00 }
%   \end{itemize}
%   比如:
%   \begin{latexexample}[gobble=6]
%     \GetIdInfo $Id$ { }
%     % 查看文件信息命令:
%     \par \ExplFileName 
%     \par \ExplFileExtension
%     \par \ExplFileVersion
%     \par \ExplFileDate 
%     \par \ExplFileDescription
%     % 将会得到:
%     [unknown]
%     [unknown extension]
%     -1
%     0000/00/00
%   \end{latexexample}
% 
%   \item 如果文件的版本号为:$-1$,那么\cs{ExplFileDate}的值将为:0000/00/00 ,比如:
%   \begin{latexexample}[gobble=6]
%     \GetIdInfo $Id: test.dtx -1 2023-09-03 11:00:00 latexstudio <latexstudio@qq.com> $ {XXX Theme for the Beamer class}
%     % 查看文件信息命令:
%     \par \ExplFileName 
%     \par \ExplFileExtension
%     \par \ExplFileVersion
%     \par \ExplFileDate 
%     \par \ExplFileDescription
%     % 将会得到:
%     test
%     dtx
%     -1
%     0000/00/00
%     XXX Theme for the Beamer class
%   \end{latexexample}
% \end{enumerate}
%
% \end{documentation}
%
% \begin{implementation}
%
% \section{\pkg{l3bootstrap}实现原理}
%
%    \begin{macrocode}
%<*package>
%<@@=kernel>
%    \end{macrocode}
%
% \subsection{\XeTeX{}中的\tn{pdfstrcmp}原语}
% 
% 只有\pdfTeX{}中有一个叫\tn{pdfstrcmp}的原语,在\XeTeX{}中仅有\tn{strcmp},
% 因此需要进行一些调整:
%    \begin{macrocode}
\begingroup\expandafter\expandafter\expandafter\endgroup
  \expandafter\ifx\csname pdfstrcmp\endcsname\relax
  \let\pdfstrcmp\strcmp
\fi
%    \end{macrocode}
%
% \subsection{\Lua{}代码的加载支持}
%
% 当使用\LuaTeX{}时,需要加载各种\Lua{}代码,代码本身在\pkg{l3luatex}中定义,
% 并被提取到一个单独的文件中,下述操作则是用于在此时及每次开始作业时加载\Lua{}代码。
%    \begin{macrocode}
\begingroup\expandafter\expandafter\expandafter\endgroup
\expandafter\ifx\csname directlua\endcsname\relax
\else
  \ifnum\luatexversion<110 %
  \else
%    \end{macrocode}
% 确保加载了对\LuaTeX{}的基本支持(这仅在plain中是必须的)。
%    \begin{macrocode}
    \begingroup\expandafter\expandafter\expandafter\endgroup
    \expandafter\ifx\csname newcatcodetable\endcsname\relax
      \input{ltluatex}%
    \fi
    \begingroup\expandafter\expandafter\expandafter\endgroup
    \expandafter\ifx\csname newluabytecode\endcsname\relax
    \else
      \newluabytecode\@expl@luadata@bytecode
    \fi
    \directlua{require("expl3")}%
%    \end{macrocode}
% 由于用户可能会创建自定义格式,故不假定只加载\Lua{}代码一次就可以匹配包模式。
% 相反,通过查询\Lua{}可以确定当前正在运行的模式。
%    \begin{macrocode}
    \ifnum 0%
      \directlua{
        if status.ini_version then
          tex.write("1")
        end
      }>0 %
      \everyjob\expandafter{%
        \the\expandafter\everyjob
        \csname\detokenize{lua_now:n}\endcsname{require("expl3")}%
      }%
    \fi
  \fi
\fi
%    \end{macrocode}
%
% \subsection{引擎要求}
%
% 代码目前需要\eTeX{},包括\tn{expanded}在内的“\pdfTeX{}扩展”集,以及Unicode
% 引擎通过展开生成任意字符记号的能力。自\TeX{} Live 2019以来,所有支持的引擎都具备
% 这些功能,因此将其作为引擎和\LaTeX{}格式支持的基准。对于\LuaTeX,需要至少
% \LuaTeX{} 5.3和有|token.set_lua|函数,这至少从\LuaTeX{} 1.10开始可用,后者也是
% \TeX{} Live 2019中的版本。(u)p\TeX{} 在\TeX{} Live 2020中才有\tn{ifincsname}
% 原语,但目前该原语在\pkg{expl3}中未使用,因此当前没有对其进行测试。
% 如果以后发生变化,则需重新审视这里的代码。
%    \begin{macrocode}
\begingroup
  \def\next{\endgroup}%
  \def\ShortText{Required primitives not found}%
  \def\LongText%
    {%
      The L3 programming layer requires the e-TeX primitives and the
      \LineBreak 'pdfTeX utilities' as described in the README file.
      \LineBreak
      These are available in the engines\LineBreak
      - pdfTeX v1.40.20\LineBreak
      - XeTeX v0.999991\LineBreak
      - LuaTeX v1.10\LineBreak
      - e-(u)pTeX v3.8.2\LineBreak
      - Prote (2021)\LineBreak
      or later.\LineBreak
      \LineBreak
    }%
  \ifnum0%
    \expandafter\ifx\csname luatexversion\endcsname\relax
      \expandafter\ifx\csname expanded\endcsname\relax\else 1\fi
    \else
      \ifnum\luatexversion<110 \else 1\fi
    \fi
    =0 %
      \newlinechar`\^^J %
      \def\LineBreak{\noexpand\MessageBreak}%
      \expandafter\ifx\csname PackageError\endcsname\relax
        \def\LineBreak{^^J}%
        \begingroup
          \lccode`\~=`\ \lccode`\}=`\ %
          \lccode`\T=`\T\lccode`\H=`\H%
          \catcode`\ =11 %
\lowercase{\endgroup\def\PackageError#1#2#3{%
\begingroup\errorcontextlines-1\immediate\write0{}\errhelp{#3}\def%
\                                                   {#1 Error: #2.^^J^^J
Type  H <return>  for immediate help}\def~{\errmessage{%
\                                                   }}~\endgroup}}%
      \fi
      \edef\next
        {%
          \noexpand\PackageError{expl3}{\ShortText}
            {\LongText Loading of expl3 will abort!}%
          \endgroup
          \noexpand\endinput
        }%
  \fi
\next
%    \end{macrocode}
%
% \subsection{\LaTeX3代码环境}  
%
% \begin{macro}{\ExplSyntaxOff}
% 在更改任何类别码之前,包模式下需在加载之前保当前情况。
% 另外注意,此处的设置意味着,一旦应用了\cs{ExplSyntaxOff},
% 它将成为一个“无操作”命令,直到使用\cs{ExplSyntaxOn}。
%    \begin{macrocode}
\protected\edef\ExplSyntaxOff
  {%
    \protected\def\noexpand\ExplSyntaxOff{}%
    \catcode   9 = \the\catcode   9\relax
    \catcode  32 = \the\catcode  32\relax
    \catcode  34 = \the\catcode  34\relax
    \catcode  58 = \the\catcode  58\relax
    \catcode  94 = \the\catcode  94\relax
    \catcode  95 = \the\catcode  95\relax
    \catcode 124 = \the\catcode 124\relax
    \catcode 126 = \the\catcode 126\relax
    \endlinechar = \the\endlinechar\relax
    \chardef\csname\detokenize{l_@@_expl_bool}\endcsname = 0\relax
  }%
%    \end{macrocode}
% \end{macro}
%
% 代码环境现在已经设置好了。
%    \begin{macrocode}
\catcode 9   = 9\relax
\catcode 32  = 9\relax
\catcode 34  = 12\relax
\catcode 58  = 11\relax
\catcode 94  = 7\relax
\catcode 95  = 11\relax
\catcode 124 = 12\relax
\catcode 126 = 10\relax
\endlinechar = 32\relax
%    \end{macrocode}
%
% \begin{variable}{\l_@@_expl_bool}
%   代码语法的状态:目前是开启的。
%    \begin{macrocode}
\chardef\l_@@_expl_bool = 1\relax
%    \end{macrocode}
%\end{variable}
%
% \begin{macro}{\ExplSyntaxOn}
% 这里的想法是,多次调用\cs{ExplSyntaxOn}不会弄乱类别码,多次调用
% \cs{ExplSyntaxOff}也不会浪费时间。另外注意,使用\cs{ExplSyntaxOn}
% 会改变\cs{ExplSyntaxOff}的定义,因此在包模式下,该函数应在加载过程结束后使用。
%    \begin{macrocode}
\protected \def \ExplSyntaxOn
  {
    \bool_if:NF \l_@@_expl_bool
      {
        \cs_set_protected:Npe \ExplSyntaxOff
          {
            \char_set_catcode:nn { 9 }   { \char_value_catcode:n { 9 } }
            \char_set_catcode:nn { 32 }  { \char_value_catcode:n { 32 } }
            \char_set_catcode:nn { 34 }  { \char_value_catcode:n { 34 } }
            \char_set_catcode:nn { 58 }  { \char_value_catcode:n { 58 } }
            \char_set_catcode:nn { 94 }  { \char_value_catcode:n { 94 } }
            \char_set_catcode:nn { 95 }  { \char_value_catcode:n { 95 } }
            \char_set_catcode:nn { 124 } { \char_value_catcode:n { 124 } }
            \char_set_catcode:nn { 126 } { \char_value_catcode:n { 126 } }
            \tex_endlinechar:D =
              \tex_the:D \tex_endlinechar:D \scan_stop:
            \bool_set_false:N \l_@@_expl_bool
            \cs_set_protected:Npn \ExplSyntaxOff { }
          }
      }
    \char_set_catcode_ignore:n           { 9 }   % tab
    \char_set_catcode_ignore:n           { 32 }  % space
    \char_set_catcode_other:n            { 34 }  % double quote
    \char_set_catcode_letter:n           { 58 }  % colon
    \char_set_catcode_math_superscript:n { 94 }  % circumflex
    \char_set_catcode_letter:n           { 95 }  % underscore
    \char_set_catcode_other:n            { 124 } % pipe
    \char_set_catcode_space:n            { 126 } % tilde
    \tex_endlinechar:D = 32 \scan_stop:
    \bool_set_true:N \l_@@_expl_bool
  }
%    \end{macrocode}
% \end{macro}
%
%    \begin{macrocode}
%</package>
%    \end{macrocode}
%
% \end{implementation}
%
% \PrintIndex
